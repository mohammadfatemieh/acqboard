\subsection{Repeated Multiply / Accumulate } 
FIR filtering is performed by the repeated multiply-accumulate module; the filter used is that described earlier. 

\subsubsection{RMAC}
\begin{figure}
\label{rmac}
\includegraphics[scale=0.7]{RMAC.svg}
\caption{Repeated Multiply-Accumulate (RMAC) module for fixed-point convolution.}
\end{figure} 
The repeated multiply-accumulate unit is composed (Figure \ref{rmac}) of the following subcomponents:

\begin{enumerate}
\item \textbf{Sample Counters}: under control of the RMAC FSM, the RMAC drives the sample buffer address pointer \signal{XA[7:0]} and the filter coefficient buffer address pointer \signal{HA[7:0]}. The sample buffer address pointer begins at location \signal{XBASE[7:0]} and counts backwards through the buffer. 
\item \textbf{Multiplier}: The pipelined multipler performs fixed-point multiplication of the input, truncating the output at 1.23 bits of data. 
\item \textbf{Extended-Resolution Accumulator} For each iteration through the sample buffer, the accumulator sums the resulting sample/coefficient products. The arithmetic is done with 7 extra bits of precision on the left side of the decimal, allowing for extended range and to prevent saturation mid-convolution. 
\item \textbf{Convergent Rounding} Convergent rounding of the input is performed, resulting in the output being in ``8.15'' format. 
\item \textbf{Overflow Detection} If the output is too large to be expressed in 1.15 format than the value saturates at either positve or negative extreme. 
\end{enumerate}


The RMAC is controlled by a FSM (Figure \ref{rmacfsm}) that is designed to convolve one channel's data per \signal{STARTMAC}. When \signal{STARTMAC} is asserted, the system convolves up to a maximum length $L$, and then asserts \signal{MACDONE}. 

\begin{figure}
\label{rmacfsm}
\includegraphics[scale=0.7]{RMAC_fsm.svg}
\caption{Controlling FSM for the RMAC.}
\end{figure}


\subsubsection{Fixed-point noise model} 
Here we develop a model of the dynamic range. \textbf{TO DO } 

\subsubsection{RMAC control}
The RMAC control (Figure \ref{rmaccontroller}) coordinates filtering
across all 10 channels as well as incrementing the base address of the
sample buffer, thus controlling the output interface of the sample
ring buffer. The associated FSM (figure \ref{rmaccontrollerfsm}) is
equally simple, asserting \signal{STARTMAC} to the RMAC engine and
waiting for completion.

\begin{figure}
\label{rmaccontroller}
\includegraphics[scale=0.7]{RMACcontrol.svg}
\caption{The RMAC controller}
\end{figure}

\begin{figure}
\label{rmaccontrollerfsm}
\includegraphics[scale=0.7]{RMACcontrol.fsm.svg}
\caption{The RMAC controller FSM}
\end{figure}
