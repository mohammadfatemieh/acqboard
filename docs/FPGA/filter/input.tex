\subsection{Input}

The FPGA input module (figure \ref{input}) controls ADC sampling, bit acquisition, offset compensation, and the eventual write-out of the sample bits. 

\begin{figure}[h!]
\begin{centering}
\label{input}
\includegraphics[scale=0.7]{input.svg}
\end{centering}
\end{figure}


\subsubsection{ADC interface}
The two sets of ADCs are serially chained, providing \signal{SDIA} and \signal{SDIB}. The ADC FSM (figure \ref{adcfsm}) controls the sampling sequence; CONCNT the delay between the assertion of \signal{CNV} and the bit read-out; \signal{BITCNT} sends the sample clock. To compesate for the ADC readout delay and the propagation delay across the galvanic isolators, we delay the \signal{LSCK} via a shift-register into \signal{BITEN}. 

\begin{figure}
\label{adcfsm}
\begin{centering}
\includegraphics[scale=0.7]{adc.inputFSM.svg}
\end{centering}
\end{figure}

We go out of our way to make sure we keep the digital signals are quite during the ADC's conversion period. 

\subsubsection{Offset arithmatic}
The 16-bit unipolar ACD samples are converted to bipolar samples and then added to the per-channel offset values. The resulting \signal{SUM} is checked for overflow and then written to. 

\subsubsection{Output writing}
For each \signal{INSAMPLE} assertion we cycle through all channels and then write the resulting offset-adjusted values to the downstream modules. 
