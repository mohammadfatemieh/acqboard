
\subsection{EEPROM I/O}

The EEPROM is a SPI-serial component which can store up to 32 kB of ram. We store 16-bit big-endian words as in table \ref{eepromaddr}. 
\begin{table}
\begin{tabular}[cc]
Word Address & Data \\
\hline
0-511 & Filter (256 2-word samples) \\
512 - 757 & sample buffer initial values (256 words) \\
1024 - 1535 & offset values for each gain (512 words) \\ 
\end{tabular}
\label{eepromaddr}
\end{table}

On each operation we execute the EEPROM's write-enable (WREN)
instruction, and then a full two bytes.  Since we have 12 bits of
address, we place the 11 input bits on the line and always have the
LSB be zero.

We use the two-byte read and two-byte write seqence capability of the
eeprom for both reads and writes. We never cross page boundary since
we always start with LSB = 0.

To interface to the SPI EEPROM we use a single output mux driven by
\signal{CNT}.

\begin{figure}[h!]
\includegraphics[scale=0.7]{EEPROMIO.svg}
\end{figure}

\begin{figure}[h!]
\includegraphics[scale=0.7]{EEPROMIO.fsm.svg}
\end{figure}

