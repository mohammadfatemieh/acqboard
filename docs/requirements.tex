
\section{Requirements}

    <para> The Acquisition Board was designed with multiunit
    electrophysiology in mind. To this end, we settled on the
    following design requirements. </para>
    
    
    <section>
      <title> Differential amplification </title>
      <para> All electrophysiological measurements are measurements of
      an electric potential -- a voltage. This voltage must be
      measured between two points, Va and Vref (the latter being a
      reference). Frequently additional sources of voltage relative to
      ground are present on both of these inputs, generated by
      uncontrolled environmental sources. This <empahsis>common-mode
      voltage </empahsis> is common to both inputs, and is ultimately
      removed by the differential amplification. </para>
      
      <para> In practice, manufacturing tolerances prevent complete
      removal of common-mode voltage. The ability of a differential
      amplifier to remove common-mode voltage is expressed as a ratio
      of... </para>
    </section>
    
    <section>
      <title> Input range, amplification, and resolution</title>
      <para> The potential voltages encountered in neural systems can
      vary over orders of magnitude. Even when measuring the same
      phenomina, differences in electrode configuration and reference
      placement can result in a tenfold difference in required
      amplification. Thus, electrophyisology amplifiers are equipped
      with programmable gain amplifiers, allowing a selection of input
      amplifications.  </para>
      
      <para> The existing equipment in MWL allows for 12-bit sampling
      of input voltages with full-scale 20 mV (gain = 1,000) to 500 uV
      (gain = 40,000). This allows for LSB sizes between 4.8 mV and
      122 nV, although system noise performance makes the latter
      number unattainable.  </para>
      
    </section>
    
    <section>
      <title> Bandwidth </title>
      <para> Electrophysiological phenomina in the brain can be losely
      divided into high-frequency, discrete events and low-frequency,
      continuous signals. Neural action potentials ("spikes") comprise
      the first category, with frequencies of interest between 1 kHz
      and 8 kHz. Electroencephalogram and local field potentials are
      believed to be the aggregate synchronous activity of large brain
      regions, and make up the bulk of the latter group. These
      slower-wave signals extend from 200 Hz down to near-DC levels. </para>

      <para> Thus any electrophysiological system will need to be able
      to isolate the higher-amplitude, low-frequency EEG from the
      low-amplitude, high-frequency spikes, and do so using controlled
      filtering that prevents saturation of analog-to-digital
      converter input. An aggregate bandwidth of 10 kHz is thus
      necessary. </para>

    </section>
    
    <section> 
      <title> Noise </title>
      
      <para> Even in the absence of any common or differential input,
      all amplifiers still generate internal electrical noise, which
      is measured relative to the input. This noise corrupts input
      signals and is generally broadband, thwarting computaitonal
      attempts at removing it. Input noise can directly contribute to
      cluster size for multiunit recordings, making unit isolation
      more difficult. </para> 

      <para> Our current amplifiers have a manufacturer-measured RTI
      noise of 20 uV peak-to-peak. This can have a substantial
      impact on recorded peak amplitude of a common 200 uV
      spike. Ideally, Soma would cut this noise amplitude down by a
      factor of 10. </para>

    </section>

    <section> 
      <title> Filtering</title>
      <para> The importance of linear phase</para>
    </section>

    <section>
      <title> Electrical Isolation </title>
    </section>

    <section>
      <title> Other criteria </title>
      <para>
	Cost, number of channels, size
      </para>
    </section>
    </section>

</article>
